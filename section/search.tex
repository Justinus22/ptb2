\documentclass[../main.tex]{subfiles}
\begin{document}

Zur Suche ist eine Frage in natürlicher Sprache gegeben.
Aus dem Fragetext wird dann mit dem selben \gls{LLM}, das für das Indizieren genutzt wurde, ein Vektor Embedding erzeugt.
Dieses Embedding wird mit den anderen Embeddings abgeglichen und die zu den ähnlichsten Embeddings gehörenden Textabschnitte werden als Suchergebnis zurückgegeben. 
Für den Abgleich wird die Kosinus-Ähnlichkeit\cite{rahutomo2012semantic} zwischen allen Vektoren im Index und dem Frage-Vektor berechnet.

\end{document}