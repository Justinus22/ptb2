\documentclass[../main.tex]{subfiles}
\begin{document}

Um den in der \nameref{chap:einleitung} erläuterten Sinn dieser Arbeit zu erfüllen, das Durchsuchen von Text beziehungweise hier Code, muss sich mit der Umsetzung der Suche beschäftigt werden, allerdings spielt sie im Gesamtkontext der Arbeit nur eine stüzende Rolle.

Nachdem eine Suchfrage formuliert ist, wird aus dem Fragetext ein Vektor Embedding erzeugt.
Dieses Embedding wird dann mit den anderen Embeddings im Index abgeglichen und die zu den ähnlichsten Embeddings gehörenden Textabschnitte werden als Suchergebnis zurückgegeben. 
Für den Abgleich wird die Kosinus-Ähnlichkeit genutzt, die ein Maß für die Ähnlichkeit von zwei Vektoren bildet \cite{rahutomo2012semantic}.

\end{document}