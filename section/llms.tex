\documentclass[../main.tex]{subfiles}
\begin{document}

\glspl{LLM} sind KI-Systeme, die darüber charakterisiert sind, dass sie Informationen über natürliche Sprachen kapseln.
Dazu gehören nebst Sprachmustern und Syntax auch die Erfassung von Bedeutung und Zusammenhängen.
\cite{zheng2023large}

Einen Ausdruck dieser Verständnisfähigkeit bringen \glspl{LLM} durch sogenannte Completions (dt. Vervollständigungen).
Dabei wird ein Textabschnitt vorgegeben und das \gls{LLM} generiert auf dessen Grundlage sukzessiv Folgetext.
In diesem Rahmen werden durch das \gls{LLM} auch sogenannte Vektor Embeddings erzeugt, welche im Folgeabschnitt erläutert werden.
\cite{naveed2023comprehensive}

Außerdem können \glspl{LLM} so konfiguriert werden, dass sie nicht deterministisch arbeiten, d. h., dass selbst bei einer gleichen Eingabe unterschiedliche Ausgaben entstehen können.
Diese Konfiguration ist bei der Arbeit mit \glspl{LLM} in diesem Bericht vorgenommen.

Die genaue technische und mathematische Umsetzung zur Funktionsweise dieser Modelle spielt dabei an dieser Stelle keine Rolle.

\end{document}