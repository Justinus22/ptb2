\documentclass[../main.tex]{subfiles}
\begin{document}

\glspl{LLM} sind \gls{KI}-Systeme, die darüber charakterisiert sind, dass sie Informationen über natürliche Sprachen erfassen können.
Dazu gehören nebst Sprachmustern und Syntax auch die Erfassung von Bedeutung und Zusammenhängen.
~\cite{zheng2023large}

Einen Ausdruck dieser Verständnisfähigkeit bringen \glspl{LLM} durch sogenannte Completions (dt. Vervollständigungen).
Dabei wird ein Textabschnitt vorgegeben und das \gls{LLM} generiert auf dessen Grundlage sukzessiv Folgetext.
~\cite{naveed2023comprehensive}

Außerdem sind \glspl{LLM} in der Regel so konfiguriert, dass sie nicht deterministisch arbeiten, d.\ h., dass selbst bei einer gleichen Eingabe unterschiedliche Ausgaben entstehen können.
Diese Konfiguration ist bei der Arbeit mit \glspl{LLM} in diesem Bericht vorgenommen.
~\cite{ouyang2023llm}

Da die technische und mathematische Funktionsweise dieser Modelle keine Relevanz für diese Arbeit hat, wird sie hier nicht weiter erläutert.

\end{document}