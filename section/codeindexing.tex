\documentclass[../main.tex]{subfiles}
\begin{document}

Der erste Schritt zum Indexing ist das Einteilen der Codebasis in Abschnitte.
Dies wird mit der Tree-Sitter Bibliothek durchgeführt, die das Aufteilen von Code bereitstellt \cite{treesitter}.
Als nächstes wird aus diesen Code Abschnitten Vektor Embeddings erzeugt und als Index gespeichert.

Zum Erfassen der Funktionstüchtigkeit wird eine Metrik eingeführt.
Es gibt ein im Anhang enthaltends Test Projekt und folgende zehn Test Fragen.
Jede dieser Fragen, die eine \enquote{Where}-Formulierungen aufweisen, zielt auf ein Codeabschnitt des Projektes ab.
\begin{table}[H]
\begin{center}
\caption{Test Fragen für Suchen im Index}
\label{tab:testfragen}
\begin{tabular}{| m {0.5cm} | m {0.9\textwidth} | }
 
 \hline
 1. & \enquote{Where is the logger implemented?}\\ 
 \hline
 2. & \enquote{Where are the setting changes for the logger handled?}\\ 
 \hline
 3. & \enquote{Where is the linting configured?}\\ 
 \hline
 4. & \enquote{Where is typescript configured?}\\ 
 \hline
 5. & \enquote{Where are the styles for my app?}\\ 
 \hline
 6. & \enquote{Where is my app created?}\\ 
 \hline
 7. & \enquote{Where do I create the IDs for my web view context?}\\ 
 \hline
 8. & \enquote{Where do I manage the state of my web view panels?}\\ 
 \hline
 9. & \enquote{Where do I configure what commands will be available in my vs code extension?}\\ 
 \hline
 10. & \enquote{Where do I retrieve information about vs code settings?}\\
 \hline 

\end{tabular}
\end{center}
\end{table}
\vspace*{-\baselineskip}

Für einen ersten \ref{test1} wird für das Projekt nach vorgestelltem Verfahren ein Index erstellt, welcher dann mit den Fragen durchsucht wird.
Enthalten die drei besten Ergnisse, also jene dessen Embedding am meisten Kosinus-Ähnlichkeit mit dem Embedding der Frage habe, den Textabschnitt, der durch die Frage gemeint ist, wird das Ergebniss als richtig gewertet.
Für richtige Ergebnisse wird die Kosinus-Ähnlichkeit multipliziert mit 100 notiert, sonst eine 0.

Vollständigkeitshalber: Alle Ergebniss wurden mit Hilfe der gpt-35-turbo \cite{completionmodel} und text-embedding-ada-002 \cite{embeddingmodel} Modelle erzeugt.
Für Completions wurde eine Temprature von 0.2 und ein Top-P-Wert von 0.3 gewählt. Alle anderen Einstellungen blieben unverändert.

\begin{table}[H]
\begin{center}
\caption{Ergebnisse der Index und Such Tests}
\label{tab:ergebnisse}
\begin{tabular}{| m {1cm} | m {1,3cm} | m {1,3cm} |m {1,3cm} |m {1,3cm} |m {1,3cm} |m {1,3cm} | m {1,3cm} |}
 \hline
 Frage & Test 1\labeltext{Test 1}{test1} & Test 2\labeltext{Test 2}{test2} & Test 3\labeltext{Test 3}{test3} & Test 4\labeltext{Test 4}{test4} & Test 5\labeltext{Test 5}{test5} & Test 6\labeltext{Test 6}{test6} & Test 7\labeltext{Test 7}{test7} \\
 \hline
 1. & 79,93 &0&79,93&79,37&80,08&79,25&80,62\\ 
 \hline
 2. & 81,72 &79,3&81,72&79,56&83,07&80,49&82,29\\ 
 \hline
 3. & 0 &0&0&0&84,36&0&0\\ 
 \hline
 4. & 80,36 &80,39&86,06&86,06&85,56&85,2&84,28\\ 
 \hline
 5. & 0 &0&0&0&81,38&75,76&79,12\\ 
 \hline
 6. & 78,16 &78,96&77,7&77,7&0&78,95&79,12\\ 
 \hline
 7. & 0 &0&80,87&0&80,87&78,44&0\\ 
 \hline
 8. & 0 &0&0&0&85,96&0&75,33\\ 
 \hline
 9. & 80,04 &0&85,4&89,28&85,16&0&0\\ 
 \hline
 10. & 75,58 &75,89&84,34&83,36&84,34&83,37&82,27\\
 \hline 

\end{tabular}
\end{center}
\end{table}
\end{document}


