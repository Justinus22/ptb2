\documentclass[../main.tex]{subfiles}
\begin{document}

Bereits als 2017 das Paper \citetitle*{vaswani2017attention} von Mitarbeitern bei Google veröffentlicht wurde, gab es Optimismus gegenüber dem vorgestellten Aufmerksamkeitsmechanismus und der Transformer Architektur, welche es ermöglichten, semantische Zusammenhänge unter anderem in Text zu erfassen.
Nach erfolgreichen Tests dieser Prinzipien mit Englisch-Deutsch-Übersetzungen wurde der Schluss gezogen: \foreignquote{english}{We are excited about the future of attention-based models and plan to apply them to other tasks. We plan to extend the Transformer to problems involving input and output modalities other than text[...]}\cite*{vaswani2017attention}.
Doch bereits die Textverarbeitung durch KI-Modelle erlebte einen Boom in der Öffentlichkeit mit dem Erscheinen von ChatGPT im November 2022.
Ein Chatbot, der die Technologie zur Textverarbeitung von natürlicher Sprache, sogenannte \glspl{LLM}, die sich den Aufmerksamkeitsmechanismus und die Transformer Architektur zu Nutze machen, leicht zugänglich präsentiert. \cite{chatgpt2023}

Durch die hohe Publizität wurde schnell nach Anwendung der Möglichkeiten in allen Bereichen des Lebens gesucht.
So wurde die Technologie im Bereich der Bildung genutzt, um mit ihr zu lernen und z. B. Ergebnisse von Übungsaufgaben abzugleichen.
Auch in der Medizin wurden Experimente mit dem Ziel durchgeführt, Patienten durch den Chatbot einen Ansprechpartner zu geben oder sogar Krankheiten durch diesen zu identifizieren.
\cite*{liu2023summary}

So kam es ebenfalls dazu, dass \glspl{LLM} Einzug in die Entwicklungsprozesse von Software gefunden haben und dort eine Reihe von Aufgaben erledigen.
Dazu gehört unter anderem die Generierung von Code, das Beheben von Fehlern im Code oder das Erklären von Code.
Empirisch konnte belegt werden, dass die Technologie teilweise besser abschneided, als es ein Mensch durchschnittlich in diesem Anwendungsbereich machen würde\cite*{tian2023chatgpt}.
Vor allem in größeren Softwareprojekten kann es schwierig werden, einen Überblick über die Codebasis zu gewinnen und zu behalten.
Deshalb wird dieser Praxistransferbericht untersuchen, wie man \glspl{LLM} nutzen kann, um Sourcecode mit Fragen in natürlicher Sprache zu durchsuchen.


\end{document}