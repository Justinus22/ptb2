\documentclass[../main.tex]{subfiles}
\begin{document}

Bei der Bewertung der Ergebnisse muss hier zuerst die genutzte Metrik bewertet werden.
Sowohl das ausgeählte Test Projekt als auch die zehn Test Fragen wurden nur beispielhaft gewählt und haben geringen repräsentativen Aussagewert, da die Ergebnisse für andere Fragen an einem anderem Projekt komplett anders ausfallen könnten.
Z. B. hätte \ref{test7} bei detailreicheren Fragen vielleicht besser abschneiden können.
Die Metrik ist dementsprechend nur bedingt aussagekräftig.

Außerdem wurden alle Tests genau einmal durchgeführt. 
Da die \glspl{LLM} allerdings nicht deterministisch Ergebnisse erzeugen, könnten auch bei einem zweiten Testlauf andere Ergebnisse entstehen.
Zudem wurde vernachlässigt, dass durch z.B. Netzwerkprobleme oder eine Limitierung der Aufrufrate von den \glspl{LLM} abgefälschte Ergebnisse entstanden sein können.
Die Ergebnisse wurden auch nur mit den zwei vorgestellten Modellen ermittelt und es lässt sich nicht verallgemeinernd auf die funktionsweise bei anderen Modellen schließen.

Insgesamt sind die Ergebnisse deshalb skeptisch zu betrachten.
Trotzdem kann unter diesen Bedingungen vermutet werden, dass das weitere Analysieren und Zusammenfassen von Code mit einem weitern Completion \gls{LLM}, welches mehr Kontext verarbeiten kann, vorteilhaft ist.
Dabei sind die Nachteile anzumerken, dass der weitere \gls{LLM} Aufruf viel Zeit verbraucht und auch zusätzliche Kosten verursacht. 

\end{document}