\documentclass[../main.tex]{subfiles}
\begin{document}

Für die Bewertung der Ergebnisse muss zunächst die genutzte Metrik bewertet werden.
Sowohl das ausgewählte Testprojekt, als auch die zehn Testfragen wurden nur beispielhaft gewählt und haben einen geringen repräsentativen Aussagewert.
Die Ergebnisse für andere Fragen an einem anderen Projekt könnten anders ausfallen.
Z. B. hätte~\ref{test7} bei detailreicheren Fragen eventuell besser abschneiden können.
Die Metrik ist dementsprechend nur bedingt aussagekräftig.

Außerdem wurden alle Tests nur genau einmal durchgeführt. 
Da die \glspl{LLM} allerdings nicht deterministisch Ergebnisse erzeugen, könnten auch bei einem zweiten Testlauf andere Ergebnisse entstehen.
Zudem wurde vernachlässigt, dass beispielsweise durch Netzwerkprobleme oder eine Limitierung der Aufrufrate von den \glspl{LLM} abgefälschte Ergebnisse entstehen können.
Die Ergebnisse wurden außerdem nur mit den zwei vorgestellten Modellen ermittelt und es lässt sich nicht verallgemeinernd auf die Funktionsweise bei anderen Modellen schließen.

Insgesamt sind die Ergebnisse deshalb kritisch zu betrachten.
Trotzdem kann die Hypothese aufgestellt werden, dass das weitere Analysieren und Zusammenfassen von Code mit einem weiteren Completion-\gls{LLM}, welches mehr Kontext verarbeiten kann, vorteilhaft ist und das Durchsuchen von Programmcode verbessert.
Dabei sind die Nachteile anzumerken, dass der weitere \gls{LLM}-Aufruf viel Zeit in Anspruch nimmt und auch zusätzliche Kosten verursacht. 

\end{document}