\documentclass[../main.tex]{subfiles}
\begin{document}

Mit einem Aufschwung von Technologien, die basierend auf \gls{KI} arbeiten, findet diese sich in vielen Anwendungsfällen wieder.
So auch in der Software-Entwicklung.
Dieser Bericht wird sich damit auseinandersetzen, wie mithilfe von \glspl{LLM} Codebasen durchsucht werden können, um dadurch einen Überblick über Projekte zu bekommen und zu behalten, um letztendlich Entwicklungsprozesse zu vereinfachen.
Dafür wird der Ansatz verfolgt, einen Index aus sogenannten Vektor Embeddings zu erstellen, welcher dann mit Fragen natürlicher Sprache dursucht werden kann.
Für das Erstellen der Vektor Embeddings werden die \glspl{LLM} genutzt.
Es werden ein Testprojekt und Testfragen herangezogen, um die im folgenden Bericht erarbeiteten Verfahren dafür zu evaluieren.

\end{document}