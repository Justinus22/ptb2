\documentclass[../main.tex]{subfiles}
\begin{document}

Um den Kontext für die \glspl{LLM} weiter zu erhöhen, wird dem Prompt in \ref{test5} der relative Dateipfad zu den Codeabschnitt mitgegeben, indem der Prompt durch den folgenden Satz ergänzt wird:
\begin{center}
\begin{lstlisting}
The file path for the snippet is "<Pfad zur Datei des Codeabschnittes>".
\end{lstlisting}
\end{center}
Aufgrund dieser Änderung wird nun auch der Dateiname übergeben, der mehr Kontext über den Zweck der Datei bereitstellen kann.
Deutlich wird dies am Beispiel von Frage 3, die nach der ESLint Konfiguration fragt.
In den vorheringen Tests konnte diese Frage nicht beantwortet werden.
Die Antwort des Completion \gls{LLM} war das vorgegebene \enquote{NOTHING}, welches nur zurückgegeben werden soll, wenn kein Code übergeben wurde.
Mit dieser Anpassung war es dem \gls{LLM} allerdings möglich, Konfigurationen richtig einzuordnen und es wurden neun von zehn Fragen richtig beantwortet.

Dabei ist auffällig, dass Frage 6 nicht beantwortet werden konnte, obwohl dies bei allen vorherigen Tests möglich war. 
Erklärbar ist dieses Verhalten dadurch, dass der Codeabschnitt selber im Wortlaut ähnlich zu der Frage ist und deshalb in den meisten Tests erkannt wird.
Doch bei \ref{test5} wird dieser Abschnitt so abstrakt beschrieben, dass die Embeddings zu fern für eine erfolgreiche Suche sind.

\end{document}