\documentclass[../main.tex]{subfiles}
\begin{document}

In \ref{test1} konnten sechs von zehn Fragen richtig beantwortet werden.
Zur weiteren Untersuchung wurde \ref{test2} durchgeführt.
In diesem Test war der Prozess des Indexing und der Suche gleich, doch wurde das Testprojekt insofern angepasst, als dass sämtliche Dokumentation gelöscht wurde und Namen von Variablen nicht deskriptiv, z. B. \enquote{logger} zu \enquote{l}, waren.
Die vorgestellte Indexing- und Suchmethode konnte dieser Änderung nicht standhalten und nur noch vier von zehn Fragen richtig beantworten.
Außerdem ist anzumerken, dass die Fragen, die richtig beantwortet wurden, jene sind, bei denen der dazugehörige Codeabschnitt weiterhin viele deskriptive Namen enthält, da dort externe Bibliotheken genutzt werden, deren Namen nicht angepasst werden können.


Aus diesen beiden Punkten lässt sich vermuten, dass bei der Erzeugung der Embeddings vor allem aus der Namensgebung Information gezogen wird und nicht aus der Logik des Codes.

\end{document}